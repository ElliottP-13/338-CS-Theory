\documentclass[11pt]{article}

%Don't change any thing before \begin{document}
%In fact if you use sth fancy, you might need
%to add more packages, or macros.


\usepackage{../EllioStyle}



\begin{document}
\date{April 3, 2020}
\ShortHeadings{Computer Science Theory: Assignment~5}{Elliott Pryor}
\title{CSCI 338: Assignment~5~(6 points)}

\author{Elliott Pryor}


\maketitle

%When writing up your solution, comment out the following until you reach Problem 1.
\noindent
This assignment is due on {\bf Monday, April 20, 11:30pm}. It is strongly
encouraged that you use Latex to generate a single pdf file and upload it
under {\em Assignment 5} on D2L. But there will NOT be a penalty for not
using Latex (to finish the assignment). This is {\bf not} a group-assignment,
so you must finish the assignment by yourself.

\section*{Problem 1}

\noindent
We are given 5 matrices $M_1,...,M_5$, their dimensions (i.e., rows by columns)
are as follows: 
$M_1$ is 15 $\times$ 20,
$M_2$ is 20 $\times$ 30,
$M_3$ is 30 $\times$ 10,
$M_4$ is 10 $\times$ 50, and
$M_5$ is 50 $\times$ 8.
\newline

\noindent
(1.1) Run the dynamic programming algorithm for {\em matrix chain multiplication} that we covered in class to produce the table $m[-,-]$.


\begin{table}[H]
    \centering
    \begin{tabular}{c|c c c c c}
       i $\setminus$ j  & 1 & 2 & 3 & 4 & 5 \\
       \hline
       1 & 0 & 9000 & 9000 & 165000 & 13600 \\
       2 & X & 0 & 6000 & 16000 & 11200 \\
       3 & X & X & 0 & 15000 & 6400 \\
       4 & X & X & X & 0 & 4000\\
       5 & X & X & X & X & 0 \\
    \end{tabular}
    \caption{Solution to 1.1}
    \label{tab:1.1}
\end{table}


\noindent
(1.2) What is the optimal solution value? Where do you find it? 

The optimal number of multiplications is 13600. It is located at m[1,5] in the top right corner of the table. 


\newpage
\section*{Problem 2}

\noindent
We are given a context-free grammar $G$ as follows:
\newline

$G$: $S\rightarrow AS|SB$

~~~  $A\rightarrow AD|DA|a$

~~~  $B\rightarrow BB|BD|b$

~~~  $D\rightarrow DD|d$.

We are also given a string $w=bdbdd$.
\newline

\noindent
(2.1) Run the dynamic programming algorithm for $A_{CFG}$ that we covered in class to produce the table $table[-,-]$.


\begin{table}[H]
    \centering
    \begin{tabular}{c|c c c c c}
       i $\setminus$ j  & 1 & 2 & 3 & 4 & 5 \\
       \hline
       1 & B & B & B & B & B \\
       2 & X & D & $\emptyset$ & $\emptyset$ & $\emptyset$ \\
       3 & X & X & B & B & B \\
       4 & X & X & X & D & D\\
       5 & X & X & X & X & D \\
    \end{tabular}
    \caption{Solution to 1.1}
    \label{tab:1.1}
\end{table}

\noindent
(2.2) How do we know whether $G$ generates $w$ from the table?


$G$ generates $w$ if S is in the top right corner (table[1,5]).

So in this case, $G$ cannot generate $w$ since $S \notin table[1,5]$

\newpage
\section*{Problem 3}

Show that $ALL_{DFA}\in$ P.


\begin{proof}


We construct a decideer for $ALL_{DFA}$ that runs in $O(n^k)$ time. We know that $ALL_{DFA}$ accepts $\sum^*$ iff we arrive at an accept state in every possible configuration. So we construct the decider $S$ as follows:

S: $<D>$ where D is some DFA:
\begin{enumerate}
\item Perform Breadth First Search on D starting at the start state $q_0$
\item Test if only accepting states were visited.
\item If the test returns true $\rightarrow$ accept\\
	Otherwise $\rightarrow$ reject
\end{enumerate}

Clearly $S$ decides $ALL_{DFA}$ as $S$ visits all possible states and if S accepts each one is an accepting state, so $L(D) = \sum^*$. If $S$ rejects then there is some non-accepting state that can be reached from $q_0$ so $L(D) \neq \sum^*$ because it can arrive at some non-accepting state.

Also, $S$ runs in polynomial time as BFS runs in polynomial time.
Therefore, we have found a decider for $ALL_{DFA}$ that runs in polynomial time. And by definition, $ALL_{DFA} \in P$ 

\end{proof}

\newpage
\section*{Problem 4}

Show that Independent Set $\in$ NP.


\begin{proof}

We construct a non-deterministic polynomial time turing machine $S$ to decide Independent Set. 

S: $<G, k>$ where G is a graph $G = (V,E)$ and $k$ is an integer st. $k \leq |V|$

\begin{enumerate}

\item Non-deterministically select some subset $c$ of $k$ verticies from $G$
\item Test whether $G$ contains any of the $k \choose 2$ edges connecting nodes in c
\item If test returns true, then $G$ has an edge connecting some $u,v \in c$ $\rightarrow$ Reject\\
	Otherwise, $G$ has no edges connecting $u, v \in c$ $\rightarrow$ Accept

\end{enumerate}

$S$ runs in $O({k \choose 2}) = O(n^2)$ time. So we have found a non-deterministic decider that runs in polynomial time. So Independent Set $\in NP$ by Theorem 7.20 

\end{proof}


\end{document}
