\documentclass[11pt]{article}

%Don't change any thing before \begin{document}
%In fact if you use sth fancy, you might need
%to add more packages, or macros.


\usepackage{../EllioStyle}



\begin{document}
\date{April 28, 2020}
\ShortHeadings{Computer Science Theory: Assignment~5}{Elliott Pryor}
\title{CSCI 338: Assignment~5~(6 points)}

\author{Elliott Pryor}


\maketitle

%When writing up your solution, comment out the following until you reach Problem 1.
\noindent
This assignment is due on {\bf Monday, April 28, 11:30pm}. It is strongly
encouraged that you use Latex to generate a single pdf file and upload it
under {\em Assignment 6} on D2L. But there will NOT be a penalty for not
using Latex (to finish the assignment). This is {\bf not} a group-assignment,
so you must finish the assignment by yourself.

\section*{Problem 1}

Problem 7.9 (page 323).
\newline


\begin{proof}


We construct a TM that decides $TRIANGLE$ in polynomial time. 

So if there is a triangle between nodes $a, b, c$ then $\exists (a, b), (a,c), (b, c) \in E$. In other words, there is an edge between each vertex. We can check if these edges exist by considering combinations of the edge set. There are ${|E| \choose 3}$ combinations of the edge set. Let $m$ be the number of edges in $G$, $m = |E|$.  Well ${m \choose 3 } = \frac{m!}{3! (m - 3)!} = \frac{m * (m-1) * (m-2)}{6} = \frac{1}{6} m^3 - 3m^2 + 2m$. Then we can iterate the list of combinations until we find one that matches the form $(a, b), (a,c), (b, c)$. 

If there is a combination that fits this form, then accept. Otherwise reject.

This runs in polynomial time since the number of combinations is polynomial with respect to the size of the edge set. So we can perform this operation in $O(m^3) = O(|E|^3)$. Therefore, we have an algorithm that decides $TRIANGLE$ in polynomial time, so $TRIANGLE \in P$ by defintion.

\end{proof}

\newpage
\section*{Problem 2}

Problem 7.21 (page 324).

Say that two Boolean formulas are equivalent if they have the same set of variables
and are true on the same set of assignments to those variables (i.e., they describe
the same Boolean function). A Boolean formula is minimal if no shorter Boolean
formula is equivalent to it. Let MIN-FORMULA be the collection of minimal
Boolean formulas. Show that if P = NP, then $MIN-FORMULA \in P$.


\begin{proof}


We must show that $MIN-FORMULA \in NP$. Then if $P = NP$ $MIN-FORMULA \in P$


We first show that given two boolean fomulae determining if they are equivalent is in NP. We construct a non-deterministic TM $S$ to decide if boolean formulae $A, B$ are equivalent. We non-deterministically select some variable assignment $x$ for the variables in $A, B$. If $A(x) \neq B(x)$ then we reject, otherwise we accept. 

$S$ runs in polynomial time as it would take $O(n)$ time to compute the truth value of $A, B$ given some assignment $x$ where $n$ is the number of variables. 


%Might need the accept state of the non-determinism on the if part%

Then we construct a non-deterministic TM $R$ to decide if a boolean formula is minimal. Given input $A$ where $A$ is a boolean formula, we non-deterministically select a boolean fomula $B$ such that $B$ is shorter than $A$. We then check if $B$ is equivalent to $A$. If $B$ is equivalent to $A$ we reject, otherwise, we accept. 

Then if $P = NP$, $R$ runs in polynomial time. Because if $P = NP$ then the problem of checking if two formulae are equivalent (TM $S$) can be deterministically solved in polynomial time. 

Therefore $MIN-FORMULA \in NP$ by definition. Because $P = NP$, $MIN-FORMULA$ must also be in P, $MIN-FORMULA \in NP$

\end{proof}


\newpage
\section*{Problem 3}

Problem 7.22 (page 324).
\newline


\newpage
\section*{Problem 4}

Problem 7.35 (page 326).


\end{document}

