\documentclass[11pt]{article}

%Don't change any thing before \begin{document}
%In fact if you use sth fancy, you might need
%to add more packages, or macros.


\usepackage{../EllioStyle}



\begin{document}
\date{April 28, 2020}
\ShortHeadings{Computer Science Theory: Assignment~6}{Elliott Pryor}
\title{CSCI 338: Assignment~6~(6 points)}

\author{Elliott Pryor}


\maketitle

%When writing up your solution, comment out the following until you reach Problem 1.
\noindent
This assignment is due on {\bf Monday, April 28, 11:30pm}. It is strongly
encouraged that you use Latex to generate a single pdf file and upload it
under {\em Assignment 6} on D2L. But there will NOT be a penalty for not
using Latex (to finish the assignment). This is {\bf not} a group-assignment,
so you must finish the assignment by yourself.

\section*{Problem 1}

Problem 7.9 (page 323).
\newline


\begin{proof}


We construct a TM that decides $TRIANGLE$ in polynomial time. 

So if there is a triangle between nodes $a, b, c$ then $\exists (a, b), (a,c), (b, c) \in E$. In other words, there is an edge between each vertex. We can check if these edges exist by considering combinations of the edge set. There are ${|E| \choose 3}$ combinations of the edge set. Let $m$ be the number of edges in $G$, $m = |E|$.  Well ${m \choose 3 } = \frac{m!}{3! (m - 3)!} = \frac{m * (m-1) * (m-2)}{6} = \frac{1}{6} m^3 - 3m^2 + 2m$. Then we can iterate the list of combinations until we find one that matches the form $(a, b), (a,c), (b, c)$. 

If there is a combination that fits this form, then accept. Otherwise reject.

This runs in polynomial time since the number of combinations is polynomial with respect to the size of the edge set. So we can perform this operation in $O(m^3) = O(|E|^3)$. Therefore, we have an algorithm that decides $TRIANGLE$ in polynomial time, so $TRIANGLE \in P$ by defintion.

\end{proof}


\newpage
\section*{Problem 2}
Problem 7.21 (page 324).

$SPATH = \{<G, a, b, k> | G$ contains a simple path of length at most $k$ from $a$ to $b\}$

$LPATH = \{G, a, b, k> | G$ contains a simple path of at least $k$ from $a$ to $b \}$


\begin{enumerate}[label=\alph*)]
\item  Show that $SPATH \in P$.

We construct a polynomial time deterministic TM $S$ to decide $SPATH$. We run Breadth first search on $G$ starting from node $a$. We run BFS from $a$ until we reach $b$. We know that the path found will be the shortest path from $a$ to $b$. Then we check if the length of this path is less than or equal to $k$. If so then we accept, otherwise we reject.

This runs in Polynomial time since it is known that BFS runs in polynomial time. 

\item Show that $LPATH \in NP-C$

We need to show that $LPATH \in NP$ and $A \leq _p LPATH$ for some NP-Complete problem $A$.

\begin{enumerate}[label=\arabic*)]

\item $LPATH \in NP$. We construct a non-deterministic TM $R$ to decide $LPATH$. We nondeterministically select some path from $a$ to $b$ in $G$. If the length of the path is greater than or equal to $k$ we accept. Otherwise we reject. 

This runs in polynomial time as we only need to compare the length of the path selected. Then by Theorem 7.20, $LPATH \in NP$


\item  We reduce the Hamiltonian Cycle problem to $LPATH$, $HAMILTON \leq _p LPATH$. We map $HAMILTONIAN$ input to input for $LPATH$ as follows. We set $k$ to be the number of nodes in $G$, $k = |V|$. We then set $a = b$ to be some node in $G$, $a \in G$.

Then if $G$ contains a Hamiltonian Cycle there is a path through $G$ that visits each node once and returns to the start node. So the length of this cycle is $|V|$. Then there is some simple path from $a$ returning to itself, of length $k$. So $<G, a, a, k> \in LPATH$.

Then if there is some simple path of length $k$ in $G$ that starts at $a$ and returns to it (eg. $<G, a, a, k> \in LPATH$ then there is a hamiltonian cycle. Because a simple path cannot visit the same vertex twice, we know that it must then visit every vertex in $G \setminus \{a\}$ exactly once, and it visits $a$ twice. So then the path is a hamiltonian circuit as it visits every vertex without repeats.

Therefore, we have reduced Hamiltonian Cycle (which is known NP in class) to $LPATH$.

\end{enumerate}

Since we have shown both 1) and 2) $LPATH \in NP-Complete$


\end{enumerate}

\newpage
\section*{Problem 2 BAD}

Problem 7.21 (page 324).

Say that two Boolean formulas are equivalent if they have the same set of variables
and are true on the same set of assignments to those variables (i.e., they describe
the same Boolean function). A Boolean formula is minimal if no shorter Boolean
formula is equivalent to it. Let MIN-FORMULA be the collection of minimal
Boolean formulas. Show that if P = NP, then $MIN-FORMULA \in P$.


\begin{proof}


We must show that $MIN-FORMULA \in NP$. Then if $P = NP$ $MIN-FORMULA \in P$


We first show that given two boolean fomulae determining if they are equivalent is in NP. We construct a non-deterministic TM $S$ to decide if boolean formulae $A, B$ are equivalent. We non-deterministically select some variable assignment $x$ for the variables in $A, B$. If $A(x) \neq B(x)$ then we reject, otherwise we accept. 

$S$ runs in polynomial time as it would take $O(n)$ time to compute the truth value of $A, B$ given some assignment $x$ where $n$ is the number of variables. 


%Might need the accept state of the non-determinism on the if part%

Then we construct a non-deterministic TM $R$ to decide if a boolean formula is minimal. Given input $A$ where $A$ is a boolean formula, we non-deterministically select a boolean fomula $B$ such that $B$ is shorter than $A$. We then check if $B$ is equivalent to $A$. If $B$ is equivalent to $A$ we reject, otherwise, we accept. 

Then if $P = NP$, $R$ runs in polynomial time. Because if $P = NP$ then the problem of checking if two formulae are equivalent (TM $S$) can be deterministically solved in polynomial time. 

Therefore $MIN-FORMULA \in NP$ by definition. Because $P = NP$, $MIN-FORMULA$ must also be in P, $MIN-FORMULA \in P$

\end{proof}


\newpage
\section*{Problem 3}

Problem 7.22 (page 324).\\
Modify the algorithm for context-free language recognition in the proof of Theorem 7.16 to give a polynomial time algorithm that produces a parse tree for a
string, given the string and a CFG, if that grammar generates the string.
\newline


We modify the algorithm from Theorem 7.16 by adding an additonal table called tree. This table stores the parse trees used to generate the substrings in each spot in table. We index the table with three variables $tree[i, j, A]$ where $i, j$ are indicies in $table$. We need the third index $A$ to identify which rule we are using, as there can be multiple rules stored at each index in $table$. 

So during initialization, along the diagonal we would create a tree at $tree[i,i, A]$

\begin{equation}
\begin{forest}
[A [$w_i$]]
\end{forest}
\end{equation}


if $A \rightarrow w_i$ is a rule.

Then during the main step, if $A \rightarrow BC$ is a rule and $table[i,k]$ contains $B$ and $table[k+1, j]$ contains $C$, we add a new tree to $tree[i, j, A]$. 

\begin{equation}
\begin{forest}
[A [{$tree[i, k, B]$}]  [{$tree[k+1, j, C]$}]]
\end{forest}
\end{equation}


We would then return $tree[1, n, S]$.

\begin{algorithm}
    \caption{Parse Tree}\label{guests}
    \begin{algorithmic}[1]
    \Function{Parse Tree}{$s, G$}
        \For{$i$ in $[1, n]$}
			\For {each variable A}
				\State Test wheterh $A \rightarrow b$ is a rule where $b = w_i$.
				\State if so, place $A$ in $table(i,i)$ and create tree (1) in $tree(i, i, A)$. 
			\EndFor
		\EndFor
		\For{$l$ in $[2 ... n]$}
			\For{i = 1 to $n - l$ + 1}
				\State Let $j = i + l - 1$
				\For{$k$ = $i$ to $j-1$}
					\For{each rule $A \rightarrow BC$}
						\If{table(i, k) contains B and table(k+1, j) contains C}
							\State put A in table(i,j)
							\State put tree (2) in tree(i,j,A)
						\EndIf	
					\EndFor				
				\EndFor			
			\EndFor
		\EndFor
       \State \textbf{return} tree(1, n, S)
    \EndFunction
    \end{algorithmic}
\end{algorithm}


\newpage
\section*{Problem 4}

Show that if P = NP, a polynomial time algorithm exists that produces a satisfying
assignment when given a satisfiable Boolean formula. (Note: The algorithm you
are asked to provide computes a function; but NP contains languages, not functions. The P = NP assumption implies that SAT is in P, so testing satisfiability is
solvable in polynomial time. But the assumption doesn’t say how this test is done,
and the test may not reveal satisfying assignments. You must show that you can find
them anyway. Hint: Use the satisfiability tester repeatedly to find the assignment
bit-by-bit.)




We can find the value assigments for SAT also in polynomial time. We iterate through the list of variables. We assign variable $x_i$ to True. We then test if the formula is satisfiable with this assignment. If it is satsifieable we move on to the next variable. If it is not satisfiable, then $x_i$ must be False because the True assignment violated a necesary constraint. So we assign it to false and move on. We do this for each variable in the set. 

\begin{algorithm}
    \caption{Parse Tree}\label{guests}
    \begin{algorithmic}[1]
    \Function{Parse Tree}{$X, f$}
        \For{$i$ in $[1, n]$}
			\State $x_i = True$
			\State \textbf{if} isSatisfiable(X, f) then continue.
			\State \textbf{else} $x_i = False$ then continue.
		\EndFor
       \State \textbf{return} $x_1 ... x_n$
    \EndFunction
    \end{algorithmic}
\end{algorithm}


\end{document}

