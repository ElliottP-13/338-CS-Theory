\documentclass[11pt]{article}

%Don't change any thing before \begin{document}
%In fact if you use sth fancy, you might need
%to add more packages, or macros.


\usepackage{../EllioStyle}



\begin{document}
\date{April 3, 2020}
\ShortHeadings{Computer Science Theory: Assignment~4}{Elliott Pryor}
\title{CSCI 338: Assignment~4~(6 points)}

\author{Elliott Pryor}


\maketitle

%When writing up your solution, comment out the following until you reach Problem 1.
\noindent
This assignment is due on {\bf Friday, April 3, 11:30pm}. It is strongly
encouraged that you use Latex to generate a single pdf file and upload it
under {\em Assignment 3} on D2L. But there will NOT be a penalty for not
using Latex (to finish the assignment). This is {\bf not} a group-assignment,
so you must finish the assignment by yourself.

\section*{Problem 1}

Let ${\cal B}$ be the set of all infinite sequences over $\{a,b\}$. Show that
${\cal B}$ is uncountable, using a proof by diagonalization.


\begin{proof} By Contradiction.


Suppose ${\cal B}$ is countable. Then its elements could be ordered $b_1, b_2, b_3, ...$

\begin{table}[H]
\centering
	\begin{tabular}{c c}
	$b_1$ =& \textbf{a}abbaa...\\
	$b_2$ =& a\textbf{b}abab...\\
	$b_3$ =& bb\textbf{b}aaa...\\
	... & \\
	\end{tabular}
\end{table}

We construct a new element $b'$ by taking the elements on the diagonal and complementing them (if they were an 'a' make it a 'b' and if they were a 'b' make it an 'a'). Clearly $b' \in {\cal B}$. So then some $b_i = b'$, but by our construction $b'$ differs from $b_i$ at the $i$th spot. So $b' \neq b_i$. A contradiction. So ${\cal B}$ is not countable.
\end{proof}


\newpage
\section*{Problem 2}

Let $T=\{(i,j,k)|i,j,k\in{\naturals}\}$. Show that $T$ is countable.


\begin{proof}


We need to show that there is some $f(x)$ that is a correspondence between $T$ and $\naturals$.  We construct this $f(x)$. 

\begin{table}[H]
\centering
	\begin{tabular}{c | c}
	$\naturals$ & $f(x)$\\ \hline
	1 & (1, 1, 1)\\
	2 & (2, 1, 1)\\
	3 & (2, 2, 1)\\
	4 & (2, 1, 2)\\
	5 & (2, 2, 2)\\
	6 & (3, 1, 1)\\
	7 & (3, 2, 1)\\
	... & ...
	\end{tabular}
\end{table}

This mapping enumerates all possible values of $T$. For a given value of $i$ it enumerates all $(j, k)$ pairs in order s.t $j+k$ is minimized and $j, k \leq i$. Once it enumerates to $i = j = k$ it increases $i$ by 1 and starts again. This is clearly one-to-one as it never lists the same point twice, and is clearly onto as it explores all possible values for $T$. Therefore we have constructed a correspondence between $T$ and $\naturals$, then $T$ must be countable.

\end{proof}


\newpage
\section*{Problem 3}

Let $INFINITE_{PDA}=\{<M>|M$ is a PDA and $L(M)$ is an infinite language$\}$.
Show that $INFINITE_{PDA}$ is decidable.
\newline




\begin{proof}

We construct a TM S for $INFINITE_{PDA}$. By Theorems 2.9 and 2.20 we can convert the PDA into a CFG $G$ in Chompsky Normal Form. We then accept if there is some derivation $D \xrightarrow{*} xDy$ where $x,y$ consist only of terminals. We reject otherwise.

\end{proof}




\newpage
\section*{Problem 4}

Let $\Sigma=\{a,b\}$. Define the following language {\em ODD}$_{TM}$:

{\em ODD}$_{TM}=\{ <M>|M$ is a TM and $L(M)$ contains only strings of odd length $\}$.

Prove that {\em ODD}$_{TM}$ is undecidable.



\begin{proof}

Let $w$ be an input $w \in L(M)$. Then we know that $A_{TM} = \{<M, w> | M$ is a TM which accepts $w \}$ is undecidable by Theorem 4.11 in class. Therefore, it is undecidable for every sentence in the language. Therefore $ODD_{TM}$ is undecidable.
\end{proof}


\newpage
\section*{Problem 5}

Show that $EQ_{CFG}$ is undecidable.
\newline

This was proven in class on Week of Mar 23-27 Undecidability. The general premise is:

$Eq_{CFG}$ = $\{ <G, H> | $ G, and H are CFG's and $L(G) = L(H) \}$. Then let set $C = (L(G) \cap \bar{L(H)}) \cup (\bar{L(G)} \cap L(H))$. 

\newpage
\section*{Problem 6}

Show that $EQ_{CFG}$ is co-Turing-recognizable.
\newline


\newpage
\section*{Problem 7}

Problem 5.3 (page 239---third edition of Sipser).
\newline

\end{document}
