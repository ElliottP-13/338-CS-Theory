\documentclass[11pt]{article}

%Don't change any thing before \begin{document}
%In fact if you use sth fancy, you might need
%to add more packages, or macros.

\usepackage{times,psfrag,epsf,epsfig,graphics}


\usepackage{../EllioStyle}



\begin{document}
\date{Feb 3, 2020}
\ShortHeadings{Computer Science Theory: Assignment~1}{Elliott Pryor}
\title{CSCI 338: Assignment~1~(6 points)}

%\author{Your Name Here}


\maketitle

%When writing up your solution, comment out the following until you reach Problem 1.
\noindent
This assignment is due on {\bf Monday, Feb 3, 11:30pm}. You will need to
use Latex to generate a single pdf file and upload it under {\em Assignment 1}
on D2L. There will be a penalty for not using Latex (to finish the assignment).
This is {\bf not} a group-assignment, so you must finish the assignment by
yourself.

\section*{Problem 1}

Prove that $1^2+3^2+5^2+\cdots+(2n-1)^2=\frac{1}{3}n(4n^2-1)$.
\newline

\begin{proof}
By Induction\\

We show that the base case of $n = 1$ holds. The left hand side of the relation is $\sum_{i=1}^1(2 * i - 1) ^2 = 1$, and the right hand side is $1/3 * 1 * (4 * 1^2 - 1) = 1$. Therefore the base case holds. 

Next we show that $n \rightarrow n + 1$. We assume that the relation holds for some $k \in \integers$. Therefore we know that $\sum_{i=1}^k  (2 * i - 1) ^2 = 1/3 * k * (4 * k^2 - 1)$. Now we show that the relation also holds for $k + 1$. 

\begin{equation*}
\begin{split}
\sum_{i=1}^{k+1}  (2 * i - 1) ^2 & = 1/3 * (k + 1) * (4 * (k+1)^2 - 1)\\
(2k + 1)^2 + \sum_{i=1}^{k}  (2 * i - 1) ^2 & = 1/3 * (k + 1) * (4 * (k^2 + 2k + 1) - 1)\\
(2k + 1)^2 + \sum_{i=1}^{k}  (2 * i - 1) ^2 & = 1/3 * (k + 1) * (4k^2 + 8k + 3)\\
(2k + 1)^2 + \sum_{i=1}^{k}  (2 * i - 1) ^2 & = 1/3 * ((4k^3 + 8k^2 + 3k) + (4k^2 + 8k + 3))\\
(4k^2 + 4k + 1) + \sum_{i=1}^{k}  (2 * i - 1) ^2 & = 1/3 * (4k^3 + 12k^2 + 12k - k + 3)\\
(4k^2 + 4k + 1) + \sum_{i=1}^{k}  (2 * i - 1) ^2 & = 1/3 * (4k^3 - k) + 1/3 * (12k^2 + 12k + 3)\\
(4k^2 + 4k + 1) + \sum_{i=1}^{k}  (2 * i - 1) ^2 & = 1/3 * k(4k^2 - 1) + (4k^2 + 4k + 1)\\
\sum_{i=1}^{k}  (2 * i - 1) ^2 & = 1/3 * k(4k^2 - 1)
\end{split}
\end{equation*}

Which is true by the inductive assumption. Therefore the inductive step and base case hold and the claim is proven true by mathematical induction.

\end{proof}

\newpage
\section*{Problem 2}

Given a planar graph $P=(V,E)$, we have Euler's formula:
$|V|+|F|-|E|=2$, where $F$ is the set of faces of $P$ and $E$ is the
set of edges of $P$.
Let $|V|=n$, where $V$ is the set of vertices of $P$.
Prove that $|F|$ is at most $2n$.
\newline

%\noindent
%{\bf Proof:} ....
%....
%$\hfill \Box$
%\newline

\newpage
\section*{Problem 3}

Prove that in any simple graph there is a path from any vertex of odd degree
to some other vertex of odd degree.
\newline

%\noindent
%{\bf Proof:} ....
%....
%$\hfill \Box$
%\newline

\newpage
\section*{Problem 4}

A fully binary tree $T$ is a tree such that all internal nodes have
two children. Prove that a fully binary tree with $n$ internal nodes
in total has $2n+1$ nodes.
\newline

%\noindent
%{\bf Proof:} ....
%....
%draw/include a figure if necessary.
%$\hfill \Box$
%\newline

\newpage.
\section*{Problem 5}

Given an undirected graph $G=(V,E)$, the breadth-first-search starting at $v\in V$
($bfs(v)$ for short) is to generate a shortest path tree starting at vertex
$v\in V$. The diameter of $G$ is the longest of all shortest paths $\delta(u,v), u,v\in V$.
\newline

When $G$ is a tree, the following algorithm is proposed to compute the
diameter of $G$.
\newline

1. Run $bfs(w), w\in V$, and compute the vertex $x\in V$ furthest from $w$.

2. Run $bfs(x)$ and compute the vertex $y\in V$ furthest from $x$.

3. Return $\delta(x,y)$ as the diameter of $G$.
\newline

Prove that this algorithm is correct; i.e., $\delta(x,y)$ is in fact the
longest among all the shortest paths between $u,v\in V$.
\newline

%\noindent
%{\bf Proof:} ....
%....
%draw/include a figure if necessary.
%$\hfill \Box$

\end{document}

